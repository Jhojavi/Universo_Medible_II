\documentclass[10pt,a4paper]{article}
\usepackage[utf8]{inputenc}
\usepackage[spanish]{babel}
\usepackage{amsmath}
\usepackage{amsfonts}
\usepackage{amssymb}
\usepackage{makeidx}
\usepackage{graphicx}
\usepackage{cite} % para contraer referencias
\usepackage{fourier}
\usepackage{xcolor}
\usepackage{hyperref}
 
\usepackage[bottom]{footmisc}
\usepackage[left=2cm,right=2cm,top=2cm,bottom=2cm]{geometry}
\title{Plan - sesión 1}


\author{\textbf{Victor M. Santos}\thanks{victorhugo\_m09@hotmail.com}, \textbf{M.Tarazona-Alvarado}\thanks{miguelta281@gmail.com}, \textbf{J. Pisco-Guabave} \thanks{jhojavi@gmail.com}. \\ Grupo Halley , \\ Universidad Industrial de Santander, Bucaramanga, Colombia.}


\date{ }


\begin{document}

\maketitle

\tableofcontents
\section{Objetivo}
Comprender las nociones fundamentales en el área de la astronomía de posición
\section{Contenido}
\begin{itemize}
\item Cuerpos celestres 
\item Astronomía de posición
\item Definiciones 
 \begin{itemize}
  \item Esfera celeste
  \item Ecuador celeste
  \item Eclíptica
  \item Cenit y nadir
  \item Coordenadas celestes
 \end{itemize}
\end{itemize}
\section{Recursos}
\begin{itemize}
 \item Salón con capacidad para 20 personas
 \item Proyector
 \item Computador
 \item Marcadores
 \item Tablero
\end{itemize}

uerpos celestes son todos aquellos objetos que forman parte del universo e interactuan con otros cuerpos mediante fuerzas gravitatorias. Los cuerpos celestes son individuales tales como: estrellas, planetas, asteroidos, etc

\section{Marco conceptual}
\subsection{Cuerpos celestes}
Se puede definir como \textit{cuerpos celestes} todos los objetos que son observables en la bóveda celeste\footnote{Esfera ideal con centro en el observador terrestre que sirve para construir una imagen simple del cielo estrellado}.

\begin{itemize}
\item \textbf{Sol:} es un astro que posee luz propia también es el centro de nuestro Sistema Solar y constituye la principal fuente calorífica y energética de éste.
\item \textbf{Luna:} es el único satélite natural de la Tierra y el quinto más grande del Sistema Solar, se cree que se origino cuando un protoplaneta del tamaño de Marte impacto la Tierra.
\item \textbf{Estrellas}: puntos luminosos centelleantes de brillos y colores diversos que giran como un conjunto en la bóveda celeste.
\item \textbf{Planetas:} puntos luminosos carentes de centello cuya posición respecto a las estrellas cambia y a diferencia de las estrellas no emiten luz propia. 
\item \textbf{Cometas:} Cuerpos que adquieren luminosidad a su paso por las inmediaciones del Sol. 
\item \textbf{Vía Láctea:} mancha blanquecina que cruza la bóveda celeste, comprende casi todo lo observable a simple vista.
\item \textbf{Nebulosas:} mancha difusa que tiene aspecto de nube. 
\end{itemize}

Otros cuerpos celestes son: las galaxias, nebulosas, novas, meteoroides y cúmulos estelares.



\end{document}