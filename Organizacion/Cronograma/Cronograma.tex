\documentclass[10pt,a4paper]{article}
\usepackage[utf8]{inputenc}
\usepackage[spanish]{babel}
\usepackage{amsmath}
\usepackage{amsfonts}
\usepackage{amssymb}
\usepackage{makeidx}
\usepackage{graphicx}
\usepackage{cite} % para contraer referencias
\usepackage{fourier}
\usepackage{xcolor}
\usepackage{hyperref}
 
\usepackage[bottom]{footmisc}
\usepackage[left=2cm,right=2cm,top=2cm,bottom=2cm]{geometry}
\title{Cronograma Universo Medible II - 2020}


\author{\textbf{Victor M. Santos}\thanks{victorhugo\_m09@hotmail.com}, \textbf{M.Tarazona-Alvarado}\thanks{miguelta281@gmail.com}, \textbf{J. Pisco-Guabave} \thanks{jhojavi@gmail.com}. \\ Grupo Halley, \\ Universidad Industrial de Santander, Bucaramanga, Colombia.}


\date{ }


\begin{document}

\maketitle

Universo medible II consta de 16 sesiones con una duración de 2 horas cada una, cuenta con material teórico y actividades didácticas. La población objetiva del proyecto son estudiantes de grado décimo y once en el sistema educativo Colombiano. 

\tableofcontents

\section{Contenidos}
\begin{itemize}
\item Nociones fundamentales (stelarium)
 \begin{itemize}
  \item Cuerpos celestes 
  \item Astronomía de posición 
  \item Definiciones (Esfera celeste, ecuador celeste, ecliptica, cenit, nadir, coordenadas celestes)
 \end{itemize}
\item Movimientos de la Tierra (achatamiento de la tierra, trompos)
 \begin{itemize}
  \item Traslación 
  \item Rotación
  \item Nutación
  \item Precesión 
 \end{itemize}
\item Distancias astronómicas 
 \begin{itemize}
 \item Unidades astronómicas
 \item Distancias en el Sistema Solar (tamaños y distancias)
 \item Paralaje (medidor de paralaje)
 \end{itemize} 
\item Solsticio y equinoccio (maqueta)
\item Constelaciones (stelarium, maqueta)
 \begin{itemize}
  \item Historia (culturas)
  \item Orientación usando constelaciones 
  \item Carta celeste y apps
 \end{itemize}
\item Propiedades de la luz (prisma, espejo, gafas, lasers)
 \begin{itemize}
 \item Reflexión
 \item Refracción
 \item Dispersión 
 \item Difracción 
 \end{itemize}
\item Coordenadas 
 \begin{itemize}
  \item Coordenadas geográficas 
  \item Coordenadas Celestes
   \begin{itemize}
    \item Coordenadas horizontales (cuadrante, sextante, instrumento de azimut)
    \item Coordenadas ecuatoriales: horarias y absolutas (esfera armilar, montura de telescopio)
    \item Coordenadas eclipticas  (esfera armila)
   \end{itemize}
 \end{itemize}
\end{itemize}


\section{Sesión 0: Bienvenida}
En esta sesión se hace una presentación general de Universo Medible II y se da una charla divulgativa sobre astronomía general. \\ 

\href{https://github.com/miguelta281/Universo_Medible_II/blob/master/Presentaciones/Bienvenida/Bienvenida.pdf}{\underline{Presentación}} \\

\href{https://github.com/miguelta281/Universo_Medible_II/blob/master/Presentaciones/Charla_bienvenida/Fetu.pdf}{\underline{Charla}} 
\section{Sesión 1: Nociones fundamentales}
En esta sesión se abarcan contenidos fundamentales para la compresión del término \textit{astronomía} como ciencia que tiene por objeto de estudio el comportamiento de los astros. Definiendo, sutilmente, características de los diferentes cuerpos que se pueden estudiar, las convenciones usadas para estudiar su movimiento aparente y las primeras interpretaciones del cosmos.\\

\href{https://github.com/miguelta281/Universo_Medible_II/blob/master/Presentaciones/Bienvenida/Bienvenida.pdf}{\underline{Presentación}} \\

\href{https://github.com/miguelta281/Universo_Medible_II/blob/master/Presentaciones/Bienvenida/Bienvenida.pdf}{\underline{Plan}} \\

\end{document}