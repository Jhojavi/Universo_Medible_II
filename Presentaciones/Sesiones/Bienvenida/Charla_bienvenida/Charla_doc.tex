\documentclass[12pt,a4paper]{article}
\usepackage[latin1]{inputenc}
\usepackage[spanish]{babel}
\usepackage{csquotes}
\usepackage{amsmath}
\usepackage{amsfonts}
\usepackage{amssymb}
\usepackage{graphicx}
\usepackage[left=2.54cm,right=2.54cm,top=2.54cm,bottom=2.54cm]{geometry}
\author{Victor M. Santos \and M.Tarazona-Alvarado}
\title{\textbf{Ponencia Universo Medible II }\\ Astronom\'ia para No Astr\'onomos \\ Los Inicios de la Astronom\'ia}

\begin{document}

\maketitle


\section*{Presentaci\'on}

El siguiente texto pretende ser gu\'ia para la presentaci\'n introductor\'ia de \textbf{Universo Medible II}. Act\'ua como un abre-bocas de la tem\'atica que se manejar\'a en sesiones posteriores. Se recomienda acompa\~nar la charla con material visual que ejemplifique las tem\'aticas del documento.

\section*{Los inicios de la astronom\'ia}  

>Es posible observar el cielo sin telescopios? >Qu\'e necesitamos para observar los astros?

\medskip 

Estas preguntas actuales son respondidas viajando al pasado, regresando a los inicios de nuestra cultura occidental, cuando la palabra \textit{mitos}\footnote{Entendida como la verdad que dec\'ia el poeta.} a\'un simbolizaba una verdad evidente para nuestros antepasados.

\medskip 

Para dar inicio a nuestra historia, es necesario situarnos en la antigua ciudad de \textit{Mileto}\footnote{Antigua ciudad griega.}, ciudad que vio crecer a un joven apasionado por el \textit{cosmos}\footnote{Entendido como el universo y la totalidad.}, su nombre era Tales. Tales nace al rededor del a\~no 624 a.C. y fallece hacia el 546 a.C. y es considerado el primer fil\'osofo occidental. Adem\'as plante\'o un teorema matem\'atico que lleva su nombre y puso las bases de una nueva verdad diferente a los \textit{mitos}.

\medskip 

Antes de la revoluci\'on iniciada por Tales, lo que conocemos ahora como \textit{astronom\'ia}\footnote{Ciencia que estudia los astros.} distaba de ser considerada una disciplina cient\'ifica. Al ser los \textit{mitos} las verdades inherentes tras la cotidianidad de la cultura griega, quienes los contaban (o cantaban) eran conocidos como \textit{poetas}, la palabra del \textit{poeta}\footnote{Los poetas eran considerados por S\'ocrates como el medio de sabidur\'ia.} era la palabra de la verdad.

\medskip 

Debido a dicha cultura m\'itica, se ten\'ian explicaciones aparentes para varios fen\'omenos tales como el d\'ia, la noche, las estrellas y los eclipses. Aunque las explicaciones, hasta ese momento correctas, eran solo cantadas por el \textit{poeta}, nadie si no \'el pod\'ia comunicar la verdad. Es as\'i que, dentro de las constelaciones greco-romanas, encontramos a \textit{Ori\'on}, el cazador, que se dispone a cazar una \textit{Liebre} junto sus dos compa\~neros de caza: el \textit{Can Mayor} y el \textit{Can Menor}. Tambi\'en encontramos historias como \textit{Ori\'on} enfrentando a \textit{Tauro}\footnote{Constelaci\'on de Taurus. En otros relatos representa a Zeus.} y \textit{Ori\'on} siendo perseguido (durante milenios) por \textit{Escorpio}.

\medskip 

Partiendo de esto, Tales inicia su aventura cuestionando las creencias tradicionales de su cultura (y de otras culturas), observando y estudiando d\'ia a d\'ia el comportamiento de los astros, de las estrellas y, en s\'intesis, del firmamento. Es en este momento, hist\'orico se podr\'ia decir, en el que iniciamos a interpretar el \textit{cosmos} como algo ajeno, independiente a las vicisitudes sobrellevadas por la humanidad y digno de ser estudiado.

\medskip 

Dadas las circunstancias, Tales de Mileto logra predecir con \'exito un eclipse e inicia con un legado que nos persigue hasta la actualidad, la \textit{astronom\'ia}. Bien se dice que una noche, Tales se encontraba apreciando y estudiando el firmamento pero, repentinamente, se vio en el fondo de un pozo, esto porque se concentraba en ver las cosas de fondo y no lo que estaba justo bajo sus pies.

\medskip 

Arist\'oles fue uno de los m\'as importantes te\'oricos de la mec\'anica celeste, para \'el los objetos en el firmamento estaban puestos all\'i, como si fueran puntos luminosos puestos en la b\'oveda celeste. Arist\'oteles planteaba que la b\'ovede se mov\'ia en un sentido, mientras la \textit{Tierra} permanec\'ia est\'atica en su centro. Debido a ese fen\'onemo se percib\'ia el movimiento de los astros desde el Este al Oeste.

\medskip 

Pero, en sus observaciones, Arist\'oteles evidenci\'o que no todo lo que lograba apreciar en el firmamento se mov\'ia de la misma forma. Hab\'ian ciertas luces que erraban del movimiento armonioso del resto de luces, adem\'as estas luces errantes brillaban de forma continua y con colores m\'as vivos que las otras. Eran los planetas m\'as cercanos a la \textit{Tierra}, mejor conocidos como Venus, Marte, J\'upiter y Saturno.

\medskip 

Fue muchos a\~nos despu\'es que astr\'onomos de la talla de Cop\'ernico, que cuestion\'o (y divulg\'o) la idea de que la \textit{Tierra} segu\'ia una \'orbita alrededor del Sol, aparecieron para vislumbrar la historia con sus magn\'ificas interpretaciones. Otro astron\'onomo de gran importancia fue Kepler, que en un principio consider\'o que el movimiento de los planetas deb\'ia cumplir ciertas leyes arm\'onicas que defin\'ian la esfericidad de los planetas como una armon\'ia.

\medskip 

Galileo fue un defensor de la teor\'ia copernicana. Implement\'o instrumentos \'opticos en sus observaciones, evidenciando fronteras m\'as lejanas para lo que nuestros ojos pod\'ian ver, reconstruyendo los l\'imites de nuestros sentidos y de nuestra capacidad para conocer. Desde 1609 construy\'o un telescopio e hizo hallazgos y observaciones trascendentales tales como:

\begin{itemize}
	\item Manchas solares.
	\item Cuatro sat\'elites de J\'upiter.\footnote{Conocidos como sat\'elites galileanos.}
	\item Fases Lunares.
	\item Monta\~nas Lunares.
	\item Fases de Venus.
\end{itemize}

Galileo tambi\'en experiment\'o con la ca\'ida libre de los cuerpos, descubriendo la \textit{Ley de la Inercia}\footnote{Conocida como la primera ley de Newton.} y otras leyes como la oscilaci\'on del p\'endulo. Tambi\'en introdujo un m\'etodo cient\'ifico a sus experimentos, m\'etodo que en la actualidad conserva sus principales caracter\'isticas.

\medskip 

 Y as\'i inici\'o la \textit{astronom\'ia}, con personas fascinadas por conocer algo m\'as, observando el cielo con el ojo descubierto e ideando herramientas para interpretar los movimientos de los astros. La astronom\'ia ha evolucionado fuertemente desde sus inicios, a lo largo de la historia tuvo aportaciones fundamentales para el entendimiento del funcionamiento del \textit{cosmos} y de la vida. Hoy podemos hacer uso de varios inventos nacidos de esta magn\'ifica ciencia, para aprender y entretenernos viendo las estrellas. Para observar los astros solo hace falta levantar nuestra mirada hacia ellos.

\begin{flushright}

\textit{Ad sidera tollere vultus}\footnote{\textit{Alza la mirada hacia los astros}. Frase de Ovidio  que hace \'enfasis en el hecho de que, mientras el resto de criaturas miran hacia el suelo, s\'olo el hombre alza la vista hacia las estrellas.}

\end{flushright}

\end{document}